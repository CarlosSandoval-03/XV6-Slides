%%%%%%%%%%%%%%%%%%%%%%%%%%%%%%%%%%%%%%%%%%%%%%%%%%%%%%%%%%%%%%%%%%%%%%%%%%%%%%%%%%%
%% This project aims to create the UNAL template for presentation.               %%
%% author:Félix Julián Gutiérrez                                                 %%
%% contacts:                                                                     %%
%%    e-mail: fjgutierrezb@unal.edu.co                                           %%
%%   www.unal.edu.co                                                             %%
%%%%%%%%%%%%%%%%%%%%%%%%%%%%%%%%%%%%%%%%%%%%%%%%%%%%%%%%%%%%%%%%%%%%%%%%%%%%%%%%%%%
\documentclass{libs/ufc_format}
% Inserting the preamble file with the packages
\input{libs/preamble.tex}
% Inserting the references file
\bibliography{references.bib}

% Title
\title[Sistemas Operativos]{\huge\textbf{}}
% Subtitle
\subtitle{\textbf{El Sistema Operativo XV6 - Capítulo 3 - Parte 2}}
% Author of the presentation
\author{Carlos Santiago Sandoval Casallas}
% Institute's Name
\institute[UNAL]{
    % email for contact
    \normalsize{\email{csandovalc@unal.edu.co}}
    \newline
    % Department Name
    \department{Departamento de Ingeniería de Sistemas e Industrial}
    \newline
    % university name
    \ufc
}
% date of the presentation
\date{\today}

%%%%%%%%%%%%%%%%%%%%%%%%%%%%%%%%%%%%%%%%%%%%%%%%%%%%%%%%%%%%%%%%%%%%%%%%%%%%%%%%%%
%% Start Document of the Presentation                                           %%               
%%%%%%%%%%%%%%%%%%%%%%%%%%%%%%%%%%%%%%%%%%%%%%%%%%%%%%%%%%%%%%%%%%%%%%%%%%%%%%%%%%
\begin{document}
% insert the code style
\input{libs/code_style}

%% ---------------------------------------------------------------------------
% First frame (with tile, subtitle, ...)
\begin{frame}
    \maketitle
\end{frame}

%% ---------------------------------------------------------------------------
% Second frame
\begin{frame}{Agenda}
    % \begin{multicols}{2}
        \tableofcontents
    % \end{multicols}
\end{frame}
%% ---------------------------------------------------------------------------
\section{Código: Asignador de memoria física}
%% ---------------------------------------------------------------------------
\section{Espacio de direcciones de un proceso}
%% ---------------------------------------------------------------------------
\section{Código: sbrk}
%% ---------------------------------------------------------------------------
\section{Código: exec}
%% ---------------------------------------------------------------------------
\section{Mundo real}

    % \begin{figure}
        % \centering
    %     \caption{Diseño del espacio de direcciones virtuales de un proceso}
    %     \includegraphics[scale=0.3]{libs/img/as.png}
    %     \source{xv6: a simple, Unix-like teaching operating system \cite{xv6_book}}
    %     \label{fig:Espacio_Direcciones}
    % \end{figure}

% %% ---------------------------------------------------------------------------
% \subsection{subsección 2}
% \begin{frame}{Bloques}
%     % Blocks styles
%     \begin{block}{Bloque azul}
%         fondo bloque en blanco.
%     \end{block}

%     \begin{alertblock}{Bloque de alerta}
%         fondo bloque en blanco.
%     \end{alertblock}

%     \begin{exampleblock}{Bloque de ejemplo}
%         fondo bloque en blanco..
%     \end{exampleblock}   
% \end{frame}

% %% ---------------------------------------------------------------------------
% \subsection{uso de cajas con enfasis}
% \begin{frame}{Para el uso con cajas, en especial programación}
%     \successbox{cajas de test}

%     \pause

%     \alertbox{Alerta de test}

%     \pause

%     \simplebox{Estado de test}
% \end{frame}

% %% ---------------------------------------------------------------------------
% \subsection{Algoritmos}
% \begin{frame}{para Algoritmos (Pseudocódigo)}
%     \begin{algorithm}[H]
%         \SetAlgoLined
%         \LinesNumbered
%         \SetKwInOut{Input}{input}
%         \SetKwInOut{Output}{output}
%         \Input{x: float, y: float}
%         \Output{r: float}
%         \While{True}{
%           r = x + y\;
%           \eIf{r >= 30}{
%            ``O valor de $r$ é maior ou iqual a 10.''\;
%            break\;
%            }{
%            ``O valor de $r$ = '', r\;
%           }
%          } 
%          \caption{Algorithm Example}
%     \end{algorithm}
% \end{frame}

% %% ---------------------------------------------------------------------------

% \begin{frame}{Insertando Algoritmos}
%     \lstset{language=Python}
%     \lstinputlisting[language=Python]{code/main.py}
% \end{frame}

% %% ---------------------------------------------------------------------------
% \begin{frame}{Insertando Algoritmos}
%     \lstinputlisting[language=C]{code/source.c}
% \end{frame}

% %% ---------------------------------------------------------------------------
% \begin{frame}{Insertando Algoritmos}
%     \lstinputlisting[language=Java]{code/helloworld.java}
% \end{frame}

% %% ---------------------------------------------------------------------------
% \begin{frame}{Insertando Algoritmos}
%     \lstinputlisting[language=HTML]{code/index.html}
% \end{frame}

% %% ---------------------------------------------------------------------------
% % This frame show an example to insert multicolumns
% \section{Sección II}
% \begin{frame}{Sección II}
%     \begin{columns}{}
%         \begin{column}{0.5\textwidth}
%             \justify
%            utilizado y justificado para 2 columnas
%         \end{column}
%         \begin{column}{0.5\textwidth}
%             \justify
%            espacioentre columnas para un segundo argumento
%         \end{column}
%     \end{columns}    
% \end{frame}

% %% ---------------------------------------------------------------------------
% % This frame show an example to insert figures
% \section{sección III}
% \begin{frame}{Sección III - Figuras}
%     \begin{figure}
%         \centering
%         \caption{logo UNAL.}
%         \includegraphics[scale=0.3]{libs/UNAL_logo.jpg}
%         \source{Obtenido del sitio oficial \cite{xv6} \cite{xv6_book}}
%         \label{fig:Logo UNAL}
%     \end{figure}
% \end{frame}

%% ---------------------------------------------------------------------------
% Reference frames
\begin{frame}[allowframebreaks]
    \frametitle{Referencias}
    \printbibliography
\end{frame}

%% ---------------------------------------------------------------------------
% Final frame
\begin{frame}
    \centering
    \huge{\textbf{\example{Gracias por la atención}}}
    
    \vspace{1cm}
    
    \Large{\textbf{Contacto:}}
    \newline
    \vspace*{0.5cm}
    \large{\email{csandovalc@unal.edu.co}}
\end{frame}

\end{document}